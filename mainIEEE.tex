


\documentclass[conference,compsoc]{IEEEtran}

\ifCLASSOPTIONcompsoc

  \usepackage[nocompress]{cite}
\else

  \usepackage{cite}
\fi

\ifCLASSINFOpdf

\else
\fi
\hyphenation{}


\begin{document}


\title{Del infinito a la computación moderna}


% author names and affiliations
% use a multiple column layout for up to three different
% affiliations
\author{\IEEEauthorblockN{Juan pablo gómez ramírez}
\IEEEauthorblockA{Faculdad de ingeniería\\Ingeniería de telecomunicaciones\\
Universidad de Antioquia\\
Medellín, Antioquia\\
Email: pablo.gomez2@udea.edu.co}
}
\maketitle





% For peer review papers, you can put extra information on the cover
% page as needed:
% \ifCLASSOPTIONpeerreview
% \begin{center} \bfseries EDICS Category: 3-BBND \end{center}
% \fi
%
% For peerreview papers, this IEEEtran command inserts a page break and
% creates the second title. It will be ignored for other modes.
\IEEEpeerreviewmaketitle



\section{Introducción}
% no \IEEEPARstart
¿Cómo nació la computación moderna? Para poder responder a esta preguntar nos tenemos que remontar al año 1874 con los postulados del matemático  alemán George Cantor los cuales tratan  sobre la teoría de conjuntos infinitos. \\
Años más tarde en la década de los 20s se llevaría a cabo una serie de discusiones que derivarían en la crisis de los fundamentos. Estas discusiones tendrían como protagonistas a Kurt Godel, Alan Turing y David Hilbert, quienes jugarían un papel muy importante en el nacimiento de la computación, aportando grandes avances a las matemáticas y computación, tales como la solución  del problema de entsheidung y la máquina de Turing por parte de Alan  Turing y Kurt Godel con sus teoremas de la incompletitud.


\section{Desarrollo}
\subsection{Sobre el infinito}
Todo comienza con la teoría de conjuntos, la cual fue postulada en el año de 1874 por el matemático alemán George Cantor, en la que postulaba que hay diferentes clases de infinitos, en los que unos podían ser más grandes que otros, tiempo después aparecieron resultados paradójicos a partir de los postulados de George Cantor, lo cual daría origen a diferentes paradojas, siendo la paradoja planteada por Bertrand Russell la que más  repercusión tendría y la cual llevaría a  discusiones de años entre diferentes bandos, pero no fue sino hasta el año 1931, que Kurt godel un joven matemático daría la estocada final con su teorema de la incompletitud y pondría el punto final a estas discusiones.
\subsection{Teorema de la incompletitud}
Para hablar del teorema de la incompletitud, debemos empezar con David Hilbert y  entender la filosofía matemática de este, la cual consistía en que todo problema matemático tiene solución, para probar esto, diseño el programa Hilbert, el cual trataba de eliminar cualquier inconsistencia de las matemáticas, creando una nueva rama de las matemáticas, la cual se llamaba matematemática. Pero todo esto fue refutado por Goldel, con su teorema, el cual logro demostrar que en cualquier sistema que contenga aritmética hay enunciados verdaderos, que dentro del sistema no se pueden probar ni negar.
\subsection{El problema de entsheidung}
Fueron estos teoremas planteados por Kurt Godel los que inspirarían a personas como Alan Turing a resolver el problema de Entsheidung el cual consistía en que se tenía que comprobar si existe algún algoritmo que decida si un problema matemático tiene o no  demostración. En la solución de este problema participaron 2 matemáticos  Alonso Church y Alan Turing, ambos lograron demostrar de manera independiente que este problema no tiene solución, para esto  el matemático Church desarrollo el concepto de algoritmo, mediante el cual Turing se basó para desarrollar posteriormente su máquina de Turing.
\subsection{Máquina de Turing}
La máquina de Turing es un concepto, una máquina teórica, la cual puede ejecutar cualquier algoritmo con instrucciones precisas  y un fin concreto, la síntesis de esta consistía en una entrada-lectura, escritura-salida.\\
El funcionamiento de esta consistía en un cabezal que puede leer o escribir sobre una cinta infinita, en esta cinta infinita, se almacenaban el algoritmo y en función de lo que este cabezal lee, almacena los datos de entrada en la cinta y con estos datos  decide cual será la próxima decisión a tomar. 
\subsection{Computación moderna}
     Gracias a la máquina de Turing se sentarían las bases de la computación moderna, pero no fue sino hasta que Jhon Von Neumann con su arquitectura de computadores  que no conoceríamos a las computadoras como lo que son hoy en día. Esta arquitectura consiste en una serie de bloques los cuales son:\\
\begin{itemize}
    \item Unidad central de procesamiento (CPU)(Encargada  de todas las operaciones de funcionamiento )
    \item Memoria principal (Es el lugar en el que se guardan los datos y softwares)
    \item Buses(comunicación entre las diferentes partes)
    \item Periféricos(teclado, mouse, pantalla)
\end{itemize}

\section{Conclusion}
Por todo esto, se inferiré que el nacimiento de la computación moderna se concibe desde la introducción del concepto del infinito y sobre este se desarrollan todos los fundamentos de la computación moderna.
\begin{thebibliography}{12}

   \bibitem{}	Du sutoy,M(2 de septiembre de 2018).George cantor el matemático que descubrió que hay muchos infinitos y no todos son del mismo tamaño. bbc. Recuperado de htpp://www.bbc.com/
        \bibitem{}Copeland,B.J(19 de junio de 2019).Alan Turing. Inglaterra Encyclopledia Britannica. Recuperado de http://Britannica.com/
        \bibitem{}Mir sabate,F(2011).La polémica intuicionismo-formalismo en los años 20.Principio del tercio excluso. Recuperado de http://www.filosofia.net/materiales/pdf23/cdm35.pdf
        \bibitem{}Bombai Gordón,F(20 de febrero de 2018).David Hilbert y la defensa del rigor matemático .ELPAÍS.Recuperado de http://www.elpais.com/
        \bibitem{}Velasco,J,J(12 de mayo de 2015).Jhon Von Neumann, el genio detrás del ordenador moderno. Eldiaro.es. Recuperado de http://eldiario.es.com/ 
        \bibitem{}Morales Medina,M,A.(23 de octubre de 2014).Que dice exactamente el primer teorema de incompletitud de Goldel[Mensaje de un blog].Recuperado de http://www.Gaussianos.com/
        \bibitem{}Peña,R.(6 de septiembre de 2012).¿Computadores de von Neumann, o computadores de Turing?.[Mensaje en un blog].Recuperado de htpp://www.blogs.elpais.com/

\end{thebibliography}




% that's all folks
\end{document}


